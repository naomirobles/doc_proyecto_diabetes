% ==================== PAQUETES ====================
\usepackage[utf8]{inputenc}
\usepackage[spanish]{babel}
\usepackage{geometry}
\geometry{top=2.5cm, bottom=2.5cm, left=2.5cm, right=2.5cm}
\usepackage{setspace}
\onehalfspacing
\usepackage{graphicx}
\usepackage{float}
\usepackage{caption}
\usepackage{subcaption}
\usepackage{amsmath}
\usepackage{amssymb}
\usepackage{algorithm}
\usepackage{algpseudocode}
\usepackage{listings}
\usepackage{xcolor}
\usepackage{hyperref}
\usepackage{booktabs}
\usepackage{multirow}
\usepackage{array}
\usepackage{longtable}
\usepackage{fancyhdr}
\usepackage{titlesec}
\usepackage[backend=biber,style=ieee,sorting=none]{biblatex}

% Configuración de Tipografía
\usepackage[T1]{fontenc}
\usepackage{helvet}
\renewcommand*\familydefault{\sfdefault}


\sloppy 
\onehalfspacing
\setlength{\parskip}{10pt}

% % Configurar encabezados
% \pagestyle{fancy}
% \fancyhf{}
% \fancyhead[LE,RO]{\small\leftmark}    % Muestra el capítulo en lugar de la sección
% \fancyhead[RE,LO]{\small\thepage}      % Número de página
% \setlength{\headheight}{13.6pt}

% Configuración de encabezados y pies de página
\pagestyle{fancy}
\fancyhf{}
\rhead{Sistema de Predicción de Diabetes}
\lhead{Cómputo Paralelo - ESCOM IPN}
\rfoot{\thepage}

% Definir palabras clave
\newcommand{\keywords}[1]{%
	\begin{center}
		\textbf{Palabras clave:} #1
	\end{center}
}

\title{TÍTULO DEL REPORTE}

\date{}

\usepackage{listings}
\usepackage{xcolor}

% \lstset{
%   language=Python,
%   backgroundcolor=\color{white},
%   basicstyle=\ttfamily\small,
%   keywordstyle=\color{blue},
%   stringstyle=\color{red!70!black},
%   commentstyle=\color{gray},
%   numbers=left,
%   numberstyle=\tiny\color{gray},
%   frame=single,
%   breaklines=true,
%   showstringspaces=false
% }




% Configuración de código
\lstset{
    language=Python,
    basicstyle=\ttfamily\small,
    keywordstyle=\color{blue}\bfseries,
    commentstyle=\color{green!60!black},
    stringstyle=\color{red},
    numbers=left,
    numberstyle=\tiny\color{gray},
    stepnumber=1,
    numbersep=10pt,
    backgroundcolor=\color{gray!10},
    showspaces=false,
    showstringspaces=false,
    showtabs=false,
    frame=single,
    tabsize=2,
    captionpos=b,
    breaklines=true,
    breakatwhitespace=false,
    escapeinside={\%*}{*)},
}
