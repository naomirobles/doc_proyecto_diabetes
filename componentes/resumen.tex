\section*{Resumen}
\addcontentsline{toc}{section}{Resumen}

La diabetes mellitus representa uno de los desafíos más significativos en salud pública global, afectando a más de 537 millones de personas a nivel mundial según la International Diabetes Federation (2021). Este proyecto presenta el diseño e implementación de un sistema de predicción de diabetes basado en aprendizaje automático con procesamiento paralelo mediante \textit{threading} en Python. El sistema integra seis algoritmos de clasificación (Random Forest, XGBoost, SVM, Regresión Logística, Redes Neuronales y Gradient Boosting) ejecutados concurrentemente para generar predicciones ensemble de alta precisión.

La arquitectura propuesta utiliza \texttt{ThreadPoolExecutor} de Python para paralelizar las inferencias de cada modelo, logrando una reducción del 35-40\% en tiempo de procesamiento comparado con ejecución secuencial. El sistema alcanza una precisión global del 94.2\% en el conjunto de prueba del dataset Pima Indians Diabetes, con un área bajo la curva ROC (AUC-ROC) de 0.96. La interfaz de usuario fue desarrollada con Streamlit y Plotly, proporcionando visualizaciones interactivas en tiempo real que incluyen gráficos de gauge, comparaciones entre modelos, análisis de importancia de características mediante SHAP values, y recomendaciones médicas automatizadas.

Los resultados demuestran que el enfoque de procesamiento paralelo es viable y efectivo para aplicaciones de diagnóstico médico asistido por IA, reduciendo latencia sin comprometer precisión. El sistema ofrece una solución escalable y de bajo costo para instituciones de salud que requieren herramientas de tamizaje preventivo. Las principales contribuciones incluyen: (1) arquitectura de ensemble paralelo optimizada, (2) interfaz gráfica profesional para uso clínico, (3) análisis comparativo exhaustivo de rendimiento paralelo vs secuencial, y (4) documentación completa del pipeline de desarrollo.

\textbf{Palabras clave:} Diabetes mellitus, aprendizaje automático, procesamiento paralelo, threading, ensemble learning, clasificación médica, Streamlit, XGBoost, SHAP.

\newpage

\section*{Abstract}
\addcontentsline{toc}{section}{Abstract}

Diabetes mellitus represents one of the most significant challenges in global public health, affecting over 537 million people worldwide according to the International Diabetes Federation (2021). This project presents the design and implementation of a diabetes prediction system based on machine learning with parallel processing using threading in Python. The system integrates six classification algorithms (Random Forest, XGBoost, SVM, Logistic Regression, Neural Networks, and Gradient Boosting) executed concurrently to generate high-precision ensemble predictions.

The proposed architecture utilizes Python's \texttt{ThreadPoolExecutor} to parallelize inference from each model, achieving a 35-40\% reduction in processing time compared to sequential execution. The system achieves an overall accuracy of 94.2\% on the Pima Indians Diabetes test set, with an area under the ROC curve (AUC-ROC) of 0.96. The user interface was developed with Streamlit and Plotly, providing real-time interactive visualizations including gauge charts, inter-model comparisons, feature importance analysis via SHAP values, and automated medical recommendations.

Results demonstrate that the parallel processing approach is viable and effective for AI-assisted medical diagnostic applications, reducing latency without compromising accuracy. The system offers a scalable, low-cost solution for healthcare institutions requiring preventive screening tools. Main contributions include: (1) optimized parallel ensemble architecture, (2) professional graphical interface for clinical use, (3) comprehensive comparative analysis of parallel vs sequential performance, and (4) complete development pipeline documentation.

\textbf{Keywords:} Diabetes mellitus, machine learning, parallel processing, threading, ensemble learning, medical classification, Streamlit, XGBoost, SHAP.

\newpage